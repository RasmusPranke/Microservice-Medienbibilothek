\documentclass{article}

\usepackage[utf8]{inputenc}
\usepackage[ngerman]{babel}

\begin{document}

Table 1: Research strategies:

Relevant types:

    Solution Proposal

     - Relevant because they are about how microservices work

    Opinion Paper

    - Relevant because they tell us why microservices are interesting

    Experience Paper

    - Relevant because they tell us about pros and cons of microservices in practice

    - Particularly important for choosing our example app

Irrelevant types:

    Validation Research:

    - Not our job

    Evaluation Research:

    - We are concerned with the state of the art now, not with its impact in the future

    Philosophical Paper:

    - We are concerned with the actualities of microservices, not their interpretation


=> Relevant studies:
    1-4
    6-8
    10-23
    25-26
    28-34
    36-45
    47-52
    54-58
    60-71

Table 2: Target problems

    Only relevant when we are looking for something very specific, otherwise not useful as an initial filter.


Table 3: Research Contribution
Relevant Types:
    Application:
        - That's our topic, more or less
    Method:
        - That's also our topic
    Problem Framing:
        - Allows us to figure out the purpose and limitations of microservices.
Irrelevant Types:
    Reference Architecture:
        - Useful as a starting point for our app. Archtiecture reuse is a useful type of reuse.
    Middleware:
        - Doesn't contribute to our research question, despite being a useful starting point for our app.
    Design Pattern:
        - Same as above
    Archtitectural language:
        - Same as above

=> Remaining Relevant studies:
    3
    7-8
    10-12
    15-17
    19-22
    25-26
    28-33
    36-37
    39-43
    47-48
    51
    54
    56-58
    61-62
    64-67
    69-71

Table 4: Research perspective:

    All aspects are relevant for our research question.


Table 5: Vertical Scope:
    To limit the scope of our paper (given that this is not even a bachelor thesis), we only consider the service layer.

=> Remaining Relevant studies:
    3
    7
    10
    15
    21-22
    33
    37
    39-40
    42
    51
    54
    56-57
    61-62
    64
    66-67
    69-71

Table 6: Software lifecycle scope:
    All stages are relevant to our research question.



Vom Titel her interessant:
    70

    71
    51
    42
    39
    19

Für App interessant:
51: Service cutter: A systematic approach to service decomposition
  Gysel M., Kolbener L., Giersche W., Zimmermann O.
Gibt uns Methode, um die Architektur anzuwenden

Für Paper interessant:

Wir extrahieren relevante Studien aus
Research on Architecting Microservices: Trends, Focus, and Potential for Industrial Adoption
  Paolo Di Francesco, Ivano Malavolta, Patricia Lago
Diese Metastuide hat nahezu, wenn nicht gar alle, Studien zu Microservices aus den 4 größten Onlinequellen extrahiert und analysiert. Damit ist die uns somit gegebene Übersicht so vollständig, wie wir es uns erhoffen können.

Durch die Methodologie der Studie, aus welcher wir diese Studien extrahiert haben, halten wir alle für relevant für den aktuellen verifizierten Stand der Technik.

67: On micro-services architecture
  Namiot, Dmitry and Sneps-Sneppe, Manfred
  sowie
19: The Design and Architecture of Microservices
  A. Sill
Wir erwarten eine grundlegende Übersicht über Microservices.

71: Towards a Technique for Extracting Microservices from Monolithic Enterprise Systems
    Levcovitz, Alessandra and Terra, Ricardo and Valente, Marco Tulio
Wir erwarten einen Einblick in die Dekomposition von Monolithischen Systemen, was einen guten Überblick über die sinvolle Anwendung von Microservices geben sollte.

39: Microservices and Their Design Trade-Offs: A Self-Adaptive Roadmap
  S. Hassan; R. Bahsoon
Wir erwarten einen Einblick in eine Methode zum Design von einzelnen Services eines Microservices-basierten Projektes und daraus folgend einen Einblick in die Anforderungen an ebenjene Services.

Diese Studien sollten uns einen sinnvollen und umfangreichen Einstieg gewährleisten, während die Liste an Studien aus unserer Grundstudie uns im weiteren Verlauf aufgeworfene Fragen beantworten können sollte.

\section{Wahl unseres Fallbeispiels: SE2-Mediathek}

Mit WAM designed: Vom Grundprinzip ähnlicher Aufbau, daher erwarten wir geringen Arbeitsaufwand.
Uns bereits bekannt, daher wenig Einarbeitung.
Dem Kurs bekannt, daher mehr Platz in der Präsentation für Inhalte
In Java geschrieben, daher einfach umzuschreiben, da wir mit Spring arbeiten sollen


\section{18.12.2018: Zusammentragen des Wissens aus den Papern}

Im Großen und Ganzen stimmen die Paper mit unserem Ursprünglichen Eindruck, dass wir die Mediathek entlang der bestehenden Services in Microservices unterteilen können, überein.
Entsprechend beginnen wir im folgenden mit dem Umschreiben der Anwendung.

Sollte dies wie erhofft gelingen können wir im Paper auf die Vorzüge des WAM-Ansatzes zum Design von Microservices eingehen. Sollten wir auf Probleme stoßen gilt es, diese zu überwinden und zu dokumentieren, wie wir dies getan haben.

\section{19.12.2018: Mediathek auf Datenbank umstellen}

Wir haben uns entschieden, die bestehende Mediathek zuerst an eine Datenbank anzuschließen, da Textdateien in keinster Weise repräsentativ für "echte" Projekte sind.

Uns ist aufgefallen, dass die Wahl der Art von Datenbank stark davon abhängt, wie wir die Microservices gestalten. Deshalb ziehen wir den Schritt, die Mediathek in MS zu zerlegen, vor.

Wir halten fest, dass wir es für sinnvoll halten, für die Monolithische Applikation eine integrierte Datenbank zu benutzen.

\subsection{Anwendung von "Towards extracting Microservices from monolithic applications"}

Wir identifizieren "kundenstamm.txt" und "medienbestand.txt" als die beiden Tabellen der Mediathek. "VerleihProtokoll.txt" dagegen stellt ledigilich den Transaktionslog einer Datenbank dar und kann damit bei unseren Überlegungen außer acht gelassen werden.

Unsere Analyse des Codes und damit das Ergebniss der Methode halten wir als Diagramm fest.

\section{08.01.2019:}

Bewertungskriterien und unsere Erwartungen an deren Veränderung mit der neuen Architekur:
\begin{description}
    \item[Funktionalität] Bestehende funktionalität muss erhalten werden.
    \item[Zuverlässigkeit] Fehlertoleranz sollte erhöht werden, durch redundanz und unabhängigkeit der einzelnen Services. Ansonsten sollte sich an der Zuverlässigkeit durch die nuéue Architektur nichts ändern.
    \item[Benutzbarkeit] Die Kommunikation der Microservices könnte eine Verzögerung in der Anfragenbearbeitung hervorrufen und somit die Bedienbarkeit beeinträchtigen.
    \item[Effizienz] Das Zeitverhalten wird sich wegen der Kommunikation verschlechtern, dahingegen sollte sich die Ressourcennutzung dank der Kapselung verbessern.
    \item[Wartbarkeit] Wir erwarten, dass die Wartbarkeit in allen Aspekten durch eine stärkere Kapselung der Komponenten steigt.
    \item[Portierbarkeit] Koexistenz und Ersetzbarkeit sollten sich verbessern, die anderen Unterkriterien sollten einen größeren Umfang erhalten, sich aber qualitativ nicht verändern.
\end{description}

\section{11.01.2019: Struktur unseres Papers}

\begin{enumerate}
    \item Einleitung
    \begin{enumerate}
        \item Insbesondere Motivation des Beispiels: Zeigen des Nutzens von Spring, beleuchten der Verbesserung der Wartbarkeit
    \end{enumerate}
    \item Was sind Microservices
    \item Welche Vor- und Nachteile haben Microservices
    \item Was ist Spring
    \item Wie unterstützt Spring das entwickeln einer Microservicebasierten Anwendung
    \item Vorstellung des Beispiels
    \begin{enumerate}
        \item Zerlegung des monolithischen Systems
        \item Implementation der neuen Architektur mithilfe von Spring
        \item Analyse der Vor- und Nachteile gegenüber der alten Architektur
    \end{enumerate}
    \item Fazit
\end{enumerate}

\end{document}
